\documentclass[12pt, titlepage]{article}

\usepackage{fullpage}
\usepackage[round]{natbib}
\usepackage{multirow}
\usepackage{booktabs}
\usepackage{tabularx}
\usepackage{graphicx}
\usepackage{float}
\usepackage{placeins}
\usepackage{hyperref}
\hypersetup{
    colorlinks,
    citecolor=black,
    filecolor=black,
    linkcolor=black,
    urlcolor=blue
}
\usepackage[utf8]{inputenc}


\newcounter{mnum}
\newcommand{\mthemnum}{M\themnum}
\newcommand{\mref}[1]{M\ref{#1}}

\title{SE 3XA3: Module Interface Specification (MIS)\\BigTwo}

\author{Team 06, Team Name: Aplus^3
		\\ Senni Tan, tans28
		\\ Manyi Cheng, chengm33
		\\ Jiaxin Tang, tangj63
}
\date{\today}

\begin{document}

\maketitle
\pagenumbering{roman}
\tableofcontents
\listoftables



\begin{table}[bp]
\caption{\bf Revision History}
\begin{tabularx}{\textwidth}{p{3cm}p{2cm}X}
\toprule {\bf Date} & {\bf Version} & {\bf Notes}\\
\midrule
Mar 16, 2021 & 0.0 & Initial Draft \\
\bottomrule
\end{tabularx}
\end{table}
\FloatBarrier

\newpage
\pagenumbering{arabic}

\section{Module Hierarchy} \label{SecMH}

This section provides an overview of the module design. Modules are summarized
in a hierarchy decomposed by secrets in Table \ref{TblMH}. The modules listed
below, which are leaves in the hierarchy tree, are the modules that will
actually be implemented.

\begin{description}
\item [\refstepcounter{mnum} \mthemnum \label{mHH}:] Hardware-Hiding Module: Hardware-Hiding Module
\item [\refstepcounter{mnum} \mthemnum \label{mHH}:] Behaviour-Hiding Module: Scene Module
\item [\refstepcounter{mnum} \mthemnum \label{mHH}:] Behaviour-Hiding Module:  Card Module
\item [\refstepcounter{mnum} \mthemnum \label{mHH}:]
Behaviour-Hiding Module:  Player Module
\item [\refstepcounter{mnum} \mthemnum \label{mHH}:] Behaviour-Hiding Module:  PlayerBot Module
\item [\refstepcounter{mnum} \mthemnum \label{mHH}:] Behaviour-Hiding Module:  Rules Module
\item [\refstepcounter{mnum} \mthemnum \label{mHH}:] Behaviour-Hiding Module: Game Module
\item [\refstepcounter{mnum} \mthemnum \label{mHH}:] Behaviour-Hiding Module:  GameplayField Module

\end{description}


\begin{table}[h!]
\centering
\begin{tabular}{p{0.3\textwidth} p{0.6\textwidth}}
\toprule
\textbf{Level 1} & \textbf{Level 2}\\
\midrule

\multirow{1}{0.3\textwidth}{Hardware-Hiding Module} & Hardware-Hiding Module \\
\midrule

\multirow{7}{0.3\textwidth}{Behaviour-Hiding Module} & Scene Module\\
& Card Module\\
& Player Module\\
& PlayerBot Module\\
& Rules Module\\
& Game Module\\
& GameplayField Module\\
\midrule

\multirow{1}{0.3\textwidth}{Software Decision Module} & N/A9No generic type)\\
\bottomrule

\end{tabular}
\caption{Module Hierarchy}
\label{TblMH}
\end{table}

\section{MIS of Scene Module}
\subsection{Uses}
Game
\subsection{Interface Syntax}
\subsubsection{Exported Types}
Scene = ?
\subsubsection{Exported Access Programs}
\begin{tabular}[pos]{|c|c|c|c|}
\hline
%	\label
\textbf{Name}& \textbf{In} & \textbf{Out} & \textbf{Exceptions} \\ \hline
 Scene & Game  & Scene  &InvalidInput \\ 
\hline
 display & - & Screen & - \\
 \hline
\end{tabular}


\subsection{Interface Semantics}
\subsubsection{State Variables}
$game$: Game

\subsubsection{Environmental Variables}
None
\subsubsection{Assumptions}
The constructor Scene is called for the object instance before any other access routine is called for that object. The constructor cannot be called on an existing object.

\subsubsection{Access Program Semantics}
Scene($game$):
\begin{itemize}
    \item transition: $game$ := $game$
    \item output: $out$ := $self$
    \item exception := $exc$ := ((typeof(game) $\neq$ Game) $\Rightarrow$ InvalidInput)
\end{itemize}

display():
\begin{itemize}
    \item output := output each component in the Game module with the expected image in the image folder to the screen.
    \item exception := None
\end{itemize}

\section{MIS of Card Module}
\subsection{Interface Syntax}
\subsubsection{Exported Types}
SuiteT = \{Spade, Heart, Club, Diamond\} \textcolor{blue}{Enum inside the Card module}\\
NumT = \{A, 2, 3, 4, 5, 6, 7, 8, 9, 10, J, Q, K\}\textcolor{blue}{Enum inside the Card module}\\
Card = ?
\subsubsection{Exported Access Programs}
\begin{tabular}[pos]{|c|c|c|c|}
\hline
%	\label
\textbf{Name}& \textbf{In} & \textbf{Out} & \textbf{Exceptions} \\ \hline
Card & SuiteT, NumT, String & Card  & InvalidInput $\lor$ FileNotFound \\ 
 \hline
rank & - & \mathbb{Z} &- \\ \hline
image & - & String &- \\ 
 \hline
\end{tabular}
\subsection{Interface Semantics}
\subsubsection{State Variables}
suit:SuiteT \\
num:NumT\\
image:String \textcolor{blue}{A string contains the location of the image file for the card}\\

\subsubsection{Environmental Variables}
None
\subsubsection{Assumptions}
The constructor Card is called for each object instance before any other access routine is called for that object. The constructor cannot be called on an existing object.

\subsubsection{Access Program Semantics}
Card($suite, num, image$):
\begin{itemize}
    \item transition: $suite, num, image$ := $suite, num, image$
    \item output: $out$ := $self$
    \item exception := $exc$ := ((typeof(suite) $\neq$ SuiteT) $\lor$  (typeof(num) $\neq$ NumT) $lor$ (typeof(image) $\neq$ String) $\Rightarrow$ InvalidInput) $\land$ (can not find file at image location $\Rightarrow$ FileNotFound)
\end{itemize}

rank():
\begin{itemize}
    \item output := $out$ := ((num == A) $\Rightarrow$ 2) $\lor$ ((num == 2) $\Rightarrow$ 1) $\lor$ ((num == 3) $\Rightarrow$ 13) $\lor$ ((num == 4) $\Rightarrow$ 12) $\lor$ ((num == 5) $\Rightarrow$ 11) $\lor$ ((num == 6) $\Rightarrow$ 10) $\lor$ ((num == 7) $\Rightarrow$ 9) $\lor$ ((num == 8) $\Rightarrow$ 8) $\lor$ ((num == 9) $\Rightarrow$ 7) $\lor$ ((num == 10) $\Rightarrow$ 6) $\lor$ ((num == J) $\Rightarrow$ 5) $\lor$ ((num == Q) $\Rightarrow$ 4) $\lor$ ((num == K) $\Rightarrow$ 3)
    \item exception := None
\end{itemize}

image()
\begin{itemize}
    \item output := $out$ := image
    \item exception := None
\end{itemize}

\section{MIS of Player Module}
\subsection{Uses}
Game
\subsection{Interface Syntax}
\subsubsection{Exported Types}
None
\subsubsection{Exported Access Programs}
{\begin{tabular}[pos]{|c|c|c|c|}
\hline
%	\label
\textbf{Name}& \textbf{In} & \textbf{Out} & \textbf{Exceptions} \\ \hline
Player & Game & Player & InvaidInput\\ 
\hline
selectCard & mouse click on screen &- &- \\ 
\hline
selectPlay & mouse click on screen & - & InvalidCombination \\ 
\hline
selectPass & mouse click on screen & - & - \\ 
\hline
NumSort & mouse click on screen & - &- \\ 
\hline
SuiteSort & mouse click on screen & - & - \\ 
\hline
Restart & mouse click on screen & - & - \\ 
\hline
Exit & mouse click on screen & - & - \\ 
\hline
\end{tabular}
}
\subsection{Interface Semantics}
\subsubsection{State Variables}
game: Game\\
selectCards: seq of Card\\
cards: seq of Card


\subsubsection{Environmental Variables}
None
\subsubsection{Assumptions}
The constructor Player is called for the object instance before any other access routine is called for that object. The constructor cannot be called on an existing object.
\subsubsection{Access Program Semantics}
Player($game$):
\begin{itemize}
    \item transition: $game, selectCards, cards$ := $game, [], assigned cards from the game$
    \item output: $out$ := $self$
    \item exception: $exc$ := ((typeof(game) $\neq$ Game) $\Rightarrow$ InvalidInput)
\end{itemize}

selectCard(screen):
\begin{itemize}
    \item transition: $selectCards$ := add the card(s) that is(are) clicked on the screen
    \item exception: None
\end{itemize}

selectPlay(screen):
\begin{itemize}
    \item transition: $selectCards, cards$ := [], remove cards in $selectCards$ from $cards[\ ]$
    \item exception: $exc$ := $\lnot$ (checkSingle($selectCards$) $\lor$ checkPair($selectCards$) $\lor$ checkFive($selectCards$) $\lor$ checkThree($selectCards$) $\lor$ checkFour($selectCards$))
    $\Rightarrow$ InvalidCombination
\end{itemize}

selectPass(screen):
\begin{itemize}
    \item transition: pass the turn for the player in the game
    \item exception: None
\end{itemize}

NumSort(screen):
\begin{itemize}
    \item transition: sort $cards[\ ]$ in the number rank order.
    \item exception: None
\end{itemize}

SuiteSort(screen):
\begin{itemize}
    \item transition: sort $cards[\ ]$ in the Suite rank order.
    \item exception: None
\end{itemize}

Restart():
\begin{itemize}
    \item transition: restart the game
    \item exception: None
\end{itemize}

Exit():
\begin{itemize}
    \item transition: exit the game and return to main menu
    \item exception: None
\end{itemize}

\subsubsection{Private Methods}
checkSingle($selectCards$):
\begin{itemize}
    \item output: $out$ := (len($selectCards$) == 1) $\Rightarrow$ True $\mid$ False
\end{itemize}

checkPair($selectCards$):
\begin{itemize}
    \item output: $out$ := ((len($selectCards$) == 2) $\land$ ($selectCards$[0] == $selectCards$[1])) $\Rightarrow$ True $\mid$ False
\end{itemize}

checkFive($selectCards$):
\begin{itemize}
    \item output: $out$ := (len($selectCards$) == 5) $\land$ (isStraight($selectCards$) $\lor$ isFlush($selectCards$) $\lor$ isFullHouse($selectCards$) $\lor$ isFullHouse2($selectCards$) $\lor$ isStraightFlush($selectCards$)) $\Rightarrow$ True $\mid$ False
\end{itemize}

checkThree($selectCards$):
\begin{itemize}
    \item output: $out$ := (len($selectCards$) == 3) $\land$ ($selectCards$[0] == $selectCards$[1]) $\land$ ($selectCards$[1] == $selectCards$[2])) $\Rightarrow$ True $\mid$ False
\end{itemize}

checkFour($selectCards$):
\begin{itemize}
    \item output: $out$ := (len($selectCards$) == 3) $\land$ ($selectCards$[0] == $selectCards$[1]) $\land$ ($selectCards$[1] == $selectCards$[2]) $\land$ ($selectCards$[2] == $selectCards$[3])) $\Rightarrow$ True $\mid$ False
\end{itemize}

isStraight($selectCards$):
\begin{itemize}
    \item output: $out$ := ((NumSort(selectCards)) $\land$ ((selectCards[0].rank() $<$ selectCards[1].rank()) $\land$ (selectCards[1].rank() $<$ selectCards[2].rank()) $\land$ (selectCards[2].rank() $<$ selectCards[3].rank()) $\land$ (selectCards[3].rank() $<$ selectCards[4].rank())) $\Rightarrow$ True $\mid$ False
\end{itemize}

isFlush($selectCards$):
\begin{itemize}
    \item output: $out$ := (selectCards[0].suite == selectCards[1].suite) $\land$ (selectCards[1].suite == selectCards[2].suite) $\land$ (selectCards[2].suite == selectCards[3].suite) $\land$ (selectCards[3].suite == selectCards[4].suite) $\Rightarrow$ True $\mid$ False
\end{itemize}

isFullHouse($selectCards$):
\begin{itemize}
    \item output: $out$ := (NumSort(selectCards) $\land$ (checkThree([selectCards[0], selectCards[1], selectCards[2]]) $\land$ checkPair([selectCards[3], selectCards[4]]) $\lor$ (checkThree([selectCards[2], selectCards[3], selectCards[4]]) $\land$ checkPair([selectCards[0], selectCards[1]])) $\Rightarrow$ True $\mid$ False
\end{itemize}

isFullHouse2($selectCards$):
\begin{itemize}
    \item output: $out$ := (NumSort(selectCards) $\land$ (checkFour([selectCards[0], selectCards[1], selectCards[2], selectCards[3]]) $\land$ checkSingle([selectCards[4]]) $\lor$ (checkFour([selectCards[1], selectCards[2], selectCards[3], selectCards[4]]) $\land$ checkSingle([selectCards[0]])) $\Rightarrow$ True $\mid$ False
\end{itemize}

isStraightFlush($selectCards$):
\begin{itemize}
    \item output: $out$ := (isStraight(selectCards) $\land$ isFlush(selectCards)) $\Rightarrow$ True $\mid$ False
\end{itemize}

\section{MIS of PlayerBot Module}
\subsection{Uses}
Card, Rules
\subsection{Interface Syntax}
\subsubsection{Exported Types}
PlayerBot = ?
\subsubsection{Exported Access Programs}
\begin{tabular}[pos]{|c|c|c|c|}
\hline
%	\label
\textbf{Name}& \textbf{In} & \textbf{Out} & \textbf{Exceptions} \\ \hline
playerBot & Sequence of Card & PlayerBot & ~ \\ \hline
playCards & Sequence of Card, Sequence of Card & ~ & ~\\ \hline
playInitTurn & ~ & ~ & ~\\ \hline
checkSingle & Sequence of Card, Sequence of Card & Card & ~ \\ \hline
checkPair & Sequence of Card, Sequence of Card & Sequence of Card & ~ \\ \hline
checkFive & Sequence of Card , Sequence of Card& Sequence of Card & ~\\ \hline
removeSet & Sequence of Card & ~ & ~ \\ \hline
passTurn & ~ & ~ & ~ \\ \hline
\end{tabular}
\subsection{Interface Semantics}
\subsubsection{State Variables}
cards: Sequence of Card // Contains all the cards owned by the current computer player\\

\noindent last: Sequence of Card// Contains all the cards played by last player
\subsubsection{Environmental Variables}

\subsubsection{Assumptions}
The constructor of playerBot is called for each instance before any access routine is called for that object. The constructor cannot be called on an existing object.
\subsubsection{Access Program Semantics} 

playerBot(s): \\
Input: A list of cards owned by the current playerBot.\\
Transition: Initialize the state variables. \\
cards := s\\
last := []\\
Output: out := self\\
Exceptions: None \\ 

\noindent playCards(s, l):\\
Input: A list of cards owned by the current playerBot. A list of cards played by the last player.\\
Transition: Check if the current player is the intial player, if it is then calls playInitTurn(), else checks the length of l.\\
If l.length == 1, calls checkSingle().\\
If l.length == 2, calls checkPair().\\
If l.length == 5, calls checkFive.\\
Let validSet := checkSingle()/checkPair()/checkFive().
If validSet == null, calls passTurn(), else\\
cards := removeSet(validSet)\\
last := validSet\\
Output: None \\
Exceptions: None \\

\noindent playInitTurn()\\
Input: None\\
Transition: Removes diamond 3 from the state variable cards and updates the state variable last.\\
cards := cards.remove(c) where c.suite == 'Diamond' $\land$ c.num == '3'\\
last := c where c.suite == 'Diamond' $\land$ c.num == '3'\\
Output: None\\
Exceptions: None \\

\noindent checkSingle(s, l):\\
Input: A list of cards owned by the current playerBot. A list of cards played by the last player.\\
Transition: None\\
Output: Valid Card to be played. \\
Exceptions: None \\

\noindent checkPair(s, l):\\
Input: A list of cards owned by the current playerBot. A list of cards played by the last player.\\
Transition: None\\
Output: Valid pair of Cards to be played. \\
Exceptions: None \\

\noindent checkFive(s, l):\\
Input: A list of cards owned by the current playerBot. A list of cards played by the last player.\\
Transition: None\\
Output: Valid combination of Cards to be played. \\
Exceptions: None \\

\noindent removeSet(validPlay):\\
Input: A list of cards owned by the current playerBot. A list of valid combination of cards.\\
Transition: cards := $\forall(c : Card|c \in validPlay : cards.remove(c))$\\
Output: None\\
Exceptions: None\\

\noindent passTurn()\\
Input: None\\
Transition: Goes to next player\\
Output: None \\
Exceptions: None \\

\section{MIS of Rules Module}
\subsection{Uses}
Card
\subsection{Interface Syntax}
\subsubsection{Exported Access Programs}
\begin{tabular}[pos]{|c|c|c|c|}
\hline
%	\label
\textbf{Name}& \textbf{In} & \textbf{Out} & \textbf{Exceptions} \\ \hline
rules & ~ & ~ & ~ \\ \hline
newDeck & ~ & ~ & ~ \\ \hline
shuffle & ~ & ~ & ~ \\ \hline
setPlayerCards & ~ & ~ & ~\\ \hline
NumSort & Sequence of Card & ~ & ~\\ \hline
SuiteSort & Sequence of Card & ~ & ~ \\ \hline
isInitPlayer & Sequence of Card & boolean & ~ \\ \hline
isVaildPair & Sequence of Card & boolean & ~ \\ \hline
isVaildStraight & Sequence of Card & boolean & ~\\ \hline
isVaildFlush & Sequence of Card & boolean & ~ \\ \hline
isValidFullHouse &  Sequence of Card & boolean & ~ \\ \hline
isValidFourOfaKind &  Sequence of Card & boolean & ~ \\ \hline
isStrongerPlay & Sequence of Card & boolean & ~ \\ \hline
isStrongerSingle & Card & boolean & ~ \\ \hline
isStrongerPair & Sequence of Card& boolean & ~ \\ \hline
isStrongerFive & Sequence of Card & boolean & ~ \\ \hline
palyCards & Sequence of Card & ~ & ~ \\ \hline
\end{tabular}
\subsection{Interface Semantics}
\subsubsection{State Constants}
Suite = \{Spade, Heart, Club, Diamond\} \\
\noindent Num = \{A, 2, 3, 4, 5, 6, 7, 8, 9, 10, J, Q, K\}
\subsubsection{State Variables}
Deck: Sequence of Card\\
\noindent Player1 : Sequence of Card\\
\noindent Player2 : Sequence of Card\\
\noindent Player3 : Sequence of Card\\
\noindent Player4 : Sequence of Card\\
\noindent last: Sequence of Card // Contains all the cards played by the last player.
\subsubsection{Assumptions}
The constructor of rules is called for each instance before any access routine is called for that object.\\

\noindent shuffle() must be called before setPlayerCards().
\subsubsection{Access Program Semantics}
rules():\\
Input: None\\
Transition: Deck := [ ]\\
Player1 := [ ]\\
Player2 := [ ]\\
Player3 := [ ]\\
Player4 := [ ]\\
last := [ ]\\
Output: out := self\\
Exceptions: None\\

\noindent newDeck():\\
Input: None\\
Transition : Deck := ($\forall$ i, j : $\mathbf{N}$, c : Card $|$ i $\in$ [0..3], j $\in$ [0..12] : Deck.add(c) where c.suite := Suite[i] $\land$ c.num := Num[j]).\\
Output: None \\
Exceptions: None \\

\noindent shuffle():\\
Input: None\\
Transition : Updates Deck by rearranging the order of Cards in Deck randomly.\\
Output: None\\
Exceptions: None\\

\noindent setPalyerCards():\\
Input: None\\
Transition: Player1 := $\forall$ i : $\mathbf{N} |$ i $\in$  [0..12] : Player1.add(Deck[i])\\
Player2 := $\forall$ i : $\mathbf{N} |$ i $\in$  [13..25] : Player2.add(Deck[i])\\
Player3 := $\forall$ i : $\mathbf{N} |$ i $\in$  [26..38] : Player3.add(Deck[i])\\
Player4 := $\forall$ i : $\mathbf{N} |$ i $\in$  [39..51] : Player4.add(Deck[i])\\
Output: None\\
Exceptions: None\\

\noindent NumSort(s):\\
Input: A list of Cards\\
Transition: None\\
Output : out := a list of Cards from s sorted in the number rank order\\
Exceptions: None\\

\noindent SuiteSort(s):\\
Input: A list of Cards\\
Transition: None\\
Output : out := a list of Cards from s sorted in the suite rank order\\
Exceptions: None\\

\noindent isInitPlayer(s):\\
Input: A list of the Cards owned by the current player.\\
Transition: None\\
Output : out := ($\exists$ i : $\mathbf{N} |$ i $\in$ [0..s.length] : s[i].suite == 'Diamond' $\land$ s[i].num == '3') $\Rightarrow$ true
Exceptions: None \\

\noindent isValidPair(s):\\
Input: A list of Cards selected by the current player.\\
Transition: None\\
Output : out := s.length == 2 $\land$ s[0].num == s[1].num\\
Exceptions: None\\

\noindent isValidStraight(s):\\
Input: A list of Cards selected by the current player.\\
Transition: None\\
Output : out := s.length == 5 $\land$ ($\forall$ i : $\mathbf{N} |$ i $\in$ [0..3] : NumSort(s)[i+1].num == NumSort(s)[i].num + 1) \\
Exceptions: None\\

\noindent isValidFlush(s):\\
Input: A list of Cards selected by the current player.\\
Transition: None\\
Output : out := s.length == 5 $\land$ ($\forall$ i : $\mathbf{N} |$ i $\in$ [1..4] : s[i].suite == s[0].suite) \\
Exceptions: None\\

\noindent isValidFullHouse(s):\\
Input: A list of Cards selected by the current player.\\
Transition: None\\
Output : out := s.length == 5 $\land$ (NumSort(s)[0] == NumSort(s)[1] $\land$ NumSort(s)[3] == NumSort(s)[4] $\land$ (NumSort(s)[2] == NumSort(s)[1] $\lor$ NumSort(s)[2] == NumSort(s)[3])) \\
Exceptions: None\\

\noindent isValidFourOfaKind(s):\\
Input: A list of Cards selected by the current player.\\
Transition: None\\
Output : out := s.length == 5 $\land$ ((NumSort(s)[0] == NumSort(s)[1] $\land$ NumSort(s)[0] == NumSort(s)[2] $\land$ NumSort(s)[0] == NumSort(s)[3]) $\lor$ (NumSort(s)[4] == NumSort(s)[1] $\land$ NumSort(s)[4] == NumSort(s)[2] $\land$ NumSort(s)[4] == NumSort(s)[3])) \\
Exceptions: None\\

\noindent isStrongerPlay(s):\\
Input: A list of Cards selected by the current player.\\
Transition: None\\
Output : $\lnot$ (s.length == last.length) $\Rightarrow$ false\\
s.length == 1 $\land$ isStrongerSingle(s[0], last[0]) $\Rightarrow$ true\\
s.length == 2 $\land$ isStrongerPair(s, last) $\Rightarrow$ true\\
s.length == 5 $\land$ isStrongerFive(s, last) $\Rightarrow$ true\\
$\lnot$ (s.length $\in$ [1, 2, 5]) $\Rightarrow$ false\\
Exceptions: None\\

\noindent isStrongerSingle(s):\\
Input: A Card selected by the current player.\\
Transition: None\\
Output : out := s.num $>$ last[0].num $\lor$ (s.suite $>$ last[0].suite $\land$ s.num == last[0].num) 
Exceptions: None\\

\noindent isStrongerPair(s):\\
Input: A list of Cards selected by the current player.\\
Transition: None\\
Output : out := isValidPair(s) $\land$ s[0].num > last[0].num $\lor$ (SuiteSort(s)[1].suite $>$ SuiteSort(last)[1].suite $\land$ s[0].num == last[0].num) 
Exceptions: None\\

\noindent isStrongerFive(s):\\
Input: A list of Cards selected by the current player.\\
Transition: None\\
Output : out := isValidFourOfaKind(s) $\land$ isValidFullhouse(last) $\Rightarrow$ true\\
isValidFourOfaKind(s) $\land$ isValidFlush(last) $\Rightarrow$ true\\
isValidFourOfaKind(s) $\land$ isValidStraight(last) $\Rightarrow$ true\\
isValidFullHouse(s) $\land$ isValidStraight(last) $\Rightarrow$ true\\
isValidFullHouse(s) $\land$ isValidFlush(last) $\Rightarrow$ true\\
isValidFullFlush(s) $\land$ isValidStraight(last) $\Rightarrow$ true\\
isValidStraight(s) $\land$ isValidStraight(last) $\land$ NumSort(s)[4].num $>$ NumSort(last)[4].num $\Rightarrow$ true\\
isValidFlush(s) $\land$ isValidFlush(last) $\land$ s[0].suite $>$ last[0].suite $\Rightarrow$ true\\
isValidFullHouse(s) $\land$ isValidFullHouse(last) $\land$ (NumSort(s)[3].num $>$ NumSort(last)[3].num) $\Rightarrow$ true\\
isValidFourOfaKind(s) $\land$ isValidFourOfaKind(last) $\land$ (NumSort(s)[4].num $>$ NumSort(last)[4].num $\lor$ NumSort(s)[0].num $>$ NumSort(last)[0].num) $\Rightarrow$ true\\
Exceptions: None\\

\noindent playCards(s):\\
Input:  A list of Cards selected by the current player.\\
Transition : isStrongerPlay(s) == ture $\Rightarrow$ last := s\\
Output: None \\
Exceptions: None \\

\section{MIS of Game Module}
\subsection{Interface Syntax}
\subsubsection{Exported Types}
Game = ?
\subsubsection{Exported Access Programs}
\begin{tabular}[pos]{|c|c|c|c|}
\hline
%	\label
\textbf{Name}& \textbf{In} & \textbf{Out} & \textbf{Exceptions} \\ \hline
Game & ~ & ~ & ~ \\ \hline
resetGame & ~ & ~ & ~ \\ \hline
playerPlayCards & sequence of Card & ~ & ~ \\ \hline
AIplayCards & ~ & ~ & ~ \\ \hline
getCardsforTurn & sequence of Card & ~ & ~ \\ \hline
updateNextTurnCards & sequence of Card & ~ & ~ \\ \hline
updateField & sequence of Card & ~ & ~ \\ \hline
updateNextTurn & ~ & ~ & ~ \\ \hline
playPassTurn & ~ & ~ & ~ \\ \hline
numberSort & ~ & ~ & ~ \\ \hline
suitSort & ~ & ~ & ~ \\ \hline
isGameOver & ~ & ~ & ~ \\ \hline
displayPassTurn & ~ & ~ & ~ \\ \hline
\end{tabular}
\subsection{Interface Semantics}
\subsubsection{State Variables}
playerCards: sequence of Cards\\
opponentLeftCards: sequence of Cards\\
opponentTopCards: sequence of Cards\\
opponentRightCards: sequence of Cards\\
playerField: sequence of Cards\\
opponentLeftField: sequence of Cards\\
opponentTopField:sequence of Cards\\
pponentRightField: sequence of Cards\\
startingTurn: boolean\\
turn: String\\ 
cardsPlayed: sequence of Cards \\
lastMove: sequence of Cards \\
lastMovePlayer: sequence of Cards \\
freeMove: boolean\\
gameOver: boolean\\




\subsubsection{Environmental Variables}

\subsubsection{Assumptions}
The constructor of Game is called for each instance before any access routine is called for that object. The constructor cannot be called on an existing object.
\subsubsection{Access Program Semantics} 

Game(): \\
Input: None.\\
Transition: Initialize the state variables for object Game: 
\begin{itemize}
    \item playerCards, opponentLeftCards, opponentTopCards, opponentRightCards := []
    \item playerField, opponentLeftField, opponentTopField, opponentRightField := []
    \item startingTurn:= true
    \item turn:= null
    \item cardsPlayed:= []
    \item lastMove:= []
    \item gameOver:= false
    \item freeMove:= false
    \item lastMovePlayer:= null
\end{itemize}\\
Output: None\\
Exceptions: None \\

\noindent resetGame()\\
Input: None\\
Transition: Resets the game to their initial states. Set seach player's deck to the randomly generated sequence of card.\\
Output: None\\
Exceptions: None \\

\noindent playerPlayCards(cards)\\
Input: None\\
Transition: playerFieldText:= "". $\neg validPlay \Rightarrow$ playerFieldText = "starting turn must be valid and contain 3 of diamonds"\\
Output: None\\
Exceptions: None \\

\noindent AIplayCards()\\
Input: None\\
Transition: Computes playableCards and update next turn.\\
Output: None\\
Exceptions: None \\

\noindent getCardsforTurn()\\
Input: None\\
Transition: None \\
Output: out := (($turn \equiv "opponentLeft" \Rightarrow$ opponentLeftCards) $\cup$ ($turn \equiv "opponentTop" \Rightarrow$ opponentTopCards) $\cup$ ($turn \equiv "opponentRight" \Rightarrow$ opponentRightCards) $\cup$ ($turn \equiv "player" \Rightarrow$ playerCards))\\
Exceptions: None \\

\noindent  updateNextTurnCards(cards)\\
Input: cards: Sequence of cards.\\
Transition: 
\begin{itemize}
                \item cardsPlayed := cardsPlayed
                \item lastMove := cards
                \item lastMovePlayer := turn
                \item freeMove :=
\end{itemize}        
Output: None\\
Exceptions: None \\

\noindent updateField(cards)\\
Input: cards: Sequence of cards.\\
Transition: ($turn \equiv "opponentLeft" \Rightarrow$ opponentLeftField := cards) $\cup$ ($turn \equiv "opponentTop" \Rightarrow$ opponentTopField := cards) $\cup$ ($turn \equiv "opponentRight" \Rightarrow$ opponentRightField := cards) $\cup$ ($turn \equiv "player" \Rightarrow$ playerField := cards)\\\\
Output: None\\
Exceptions: None \\

\noindent updateNextTurn()  \\
Input: None\\
Transition: ($turn \equiv "opponentLeft" \Rightarrow$ turn := "player") $\cup$ ($turn \equiv "opponentTop" \Rightarrow$ turn := "opponentLeft") $\cup$ ($turn \equiv "opponentRight" \Rightarrow$ turn := "opponentRight") $\cup$ ($turn \equiv "player" \Rightarrow$ turn := "opponentRight")\\
Output: None\\
Exceptions: None \\

\noindent playerPassTurn()\\
Input: None\\
Transition: ($startingTurn \Rightarrow$ playerFieldText := "You cannot pass the first turn") $\cup$ ($\neg startingTurn  \Rightarrow$ playerFieldText := "")\\
Output: None\\
Exceptions: None \\


\noindent numberSort()\\
Input: None\\
Transition: playerCards := playerCards.sortCardsValue()\\
Output: None\\
Exceptions: None \\


\noindent suitSort()\\
Input: None\\
Transition:playerCards := playerCards.sortCardsSuit() \\
Output: None\\
Exceptions: None \\

\noindent isGameOver() \\
Input: None\\
Transition: ($ len(currentPlayerCards) \equiv 0 \Rightarrow$ gameOver := true\\
Output: out := ($ len(currentPlayerCards) \equiv 0 \Rightarrow$ true)\\
Exceptions: None \\

\noindent displayPassTurn() \\
Input: None\\
Transition: Display a text to the user to indicate pass turn. \\
Output: None\\
Exceptions: None \\


\section{MIS of gameplayField Module}
\subsection{Uses}
Game
\subsection{Interface Syntax}
\subsubsection{Exported Types}
\subsubsection{Exported Access Programs}
\begin{tabular}[pos]{|c|c|c|c|}
\hline
%	\label
\textbf{Name}& \textbf{In} & \textbf{Out} & \textbf{Exceptions} \\ \hline
 render & ~  & HTML Card  &~ \\ 
\hline
\end{tabular}


\subsection{Interface Semantics}
\subsubsection{State Variables}

\subsubsection{Environmental Variables}
None
\subsubsection{Assumptions}

\subsubsection{Access Program Semantics}

\noindent render()\\
Input: None\\
Transition: None\\
Output: Arrangement of players in gameplay field. \\
Exceptions: None \\
\end{document}